\documentclass[12pt, a4paper]{article}
\usepackage[francais]{babel}
\usepackage{caption}
\usepackage{graphicx}
\usepackage[T1]{fontenc}
\usepackage{listings}
\usepackage{geometry}
\usepackage{minted}
\usepackage{array,multirow,makecell}
\usepackage[colorlinks=true,linkcolor=black,anchorcolor=black,citecolor=black,filecolor=black,menucolor=black,runcolor=black,urlcolor=black]{hyperref}
\setcellgapes{1pt}
\makegapedcells
\usepackage{fancyhdr}
\pagestyle{fancy}
\lhead{}
\rhead{}
\chead{}
\rfoot{\thepage}
\lfoot{Martin Baumgaertner - Mikhaïl Karapetyan}
\cfoot{}
\renewcommand{\footrulewidth}{0.4pt}
\renewcommand{\headrulewidth}{0.4pt}
\renewcommand{\listingscaption}{Code}
\renewcommand{\listoflistingscaption}{Table des codes}
% \usepackage{mathpazo} --> Police à utiliser lors de rapports plus sérieux

\begin{document}
\begin{titlepage}
	\newcommand{\HRule}{\rule{\linewidth}{0.5mm}} 
	\center 
	\textsc{\LARGE iut de colmar}\\[6.5cm] 
	\textsc{\Large TP 2 - tp wifi avancé sous gnu/linux}\\[0.5cm] 
	\textsc{\large Année 2022-23}\\[0.5cm]
	\HRule\\[0.75cm]
	{\huge\bfseries R301 - Réseaux de campus}\\[0.4cm]
	\HRule\\[1.5cm]
	\textsc{\large martin baumgaertner - mikhaïl karapetyan}\\[6.5cm] 

	\vfill\vfill\vfill
	{\large\today} 
	\vfill
\end{titlepage}
\newpage
\tableofcontents
\listoffigures
\listoflistings
\newpage

\section{Connexion par pont}
\subsection{Dans quel dossier se trouvent les fichiers d'exemples par défaut de hostapd ?}

Les dossiers des fichiers se trouvent dans \textit{etc/default/hostapd}.

\subsection{Depuis le smartphone, notez votre masque/IP de la passerelle et les DNS reçus}
Voici les différents éléments que nous avons reçus :\\
\begin{itemize}
    \item \texttt{Adresse IP} : 10.129.11.101
    \item \texttt{Masque} : 255.255.254.0
    \item \texttt{IP de la passerelle} : 10.129.10.1
    \item \texttt{DNS reçus} : 10.252.4.42/10.252.4.43 
\end{itemize}



\section{Passerelle connectée à un point d'accès}

\subsection{Relevez manuellement les adresses IP des serveurs DNS de l'UHA}
\begin{itemize}
    \item \texttt{DNS 1} : 10.252.4.42
    \item \texttt{DNS 2} : 10.252.4.43 
\end{itemize}

\subsection{Cherchez les options les sécurisation des transactions DNS}

    
\end{document}